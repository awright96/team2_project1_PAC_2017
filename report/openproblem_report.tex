\documentclass{article}
\usepackage{algorithm, algpseudocode}
\usepackage{amsmath, amssymb, amsthm}
\usepackage{color}
\usepackage{enumerate}
\usepackage{geometry}
\usepackage{graphicx}
\usepackage{hyperref}
\usepackage{fancyhdr,lastpage}
\usepackage{url}
\usepackage{tabularx}
\pagestyle{fancy}

% Create theorem environments as needed
\newtheorem{theorem}{Theorem}
\newtheorem{definition}{Definition}

\title{Team 2 Opportunity Identification Project}
\author{Lingnan Gao, Andrew Wright}
\date{October 2017}


\begin{document}
\maketitle
\begin{abstract}
In this report, we identify the problem of Makespan Minimized Scheduling with
Dependencies as a problem ripe for a Fixed Parameter Tractable (FPT) approach. While little work has been
done in the area of FPT algorithms in operations research, recent work on the
related problem of Makespan Minimized Scheduling without dependencies
~\cite{mnich2015scheduling} demonstrates that there is promise to make progress
on the topic.
\end{abstract}
\newcommand{\defproblem}[4]{%
  \hfill\\\smallskip\noindent%
  \begin{tabularx}{\textwidth}{|l X|}%
    \hline%
    \multicolumn{2}{|l|}{{#1}}\\%
    \textbf{Input:}&#2\\%
%    \textbf{Parameter:}&#3\\%
    \textbf{Question:}&#4\smallskip\\\hline%
  \end{tabularx}%
  \smallskip%
}%
\section{Introduction}
Scheduling on Directed Acyclic Graphs (DAG) is a crucial problem with many
applications such as in the cases of task scheduling in parallel
systems and workflow distributions. Typically, an optimal scheduling
on a DAG would lead to a minimized completion time,
reducing costs, etc.. Due to the significance of this problem, it has been
extensively studied; challenges arise when it comes to find an optimal solution
to this problem, as this problem is NP-hard. Finding an optimal solution using
current methods will
more likely than not
be computationally prohibitive to find.

Although much work has been carried out to find an solution to this problem, most
of this work relies on heuristic methods (e.g. HEFT \cite{topcuoglu2002performance}) or
meta-heuristics (e.g. generic algorithms) to solve this problem.
While these problems may work well under certain conditions,
they do not provide an efficient, general solution.

One possible alternative solution to this problem is to develop a Fixed
parameter Tractable (FPT)
algorithm to solve this problem. However, to the best of our knowledge,
limited work has been done in the area of FPT algorithms in scheduling
research, and no work has been done on the problem of developing FPT algorithms
for dependency scheduling problems.

\section{Open Problem Description}
The open problem we have identified is makespan scheduling on a DAG.
Given a weighted Directed Acyclic Graph, $G=(V,E)$, the vertex
$v \in V$ denotes a task to be executed, while the weight on $v$
represents the execution time. A directed edge on $G$ reflects
dependencies among tasks, an edge from vertex $u$ to $v$
means task $v$ cannot be started unless $u$ has been completed.
Our objective is, given M identical machines to process tasks on, to find out the execution
orders for those tasks in order minimize the time to completion of all tasks.


\begin{definition}
The \emph{makespan} of a schedule of tasks $S$ is the total time to completion
from the start of the first task to the end of the last task.
\end{definition}


\defproblem{Makespan scheduling on Directed Acyclic Graph}%
{A directed graph $G = \{V, E\}$ with a function $W_v: V \to \mathbb{R}$,
 where $W_v$ is vertex weights, and a set of identical machines,
$M$}%
{}%
{Is there a Fixed Parameter Tractable algorithm parameterized by tree width
which  minimizes the makespan while respecting
dependencies?}%

A possible parameter for development of an FPT algorithm might be the number of
execution machines $M$. Another parameter which makes some intuitive sense may
be the tree width of the graph.

\section{Related Questions Works}
To the best our knowledge, the FPT hardness result and solutions to the
makespan scheduling for DAG still remains open.
However, we observe that this problems is closely related to the problem of
makespan scheduling without dependencies.

The makespan scheduling problem states that for $n$ weighted tasks with no
dependencies among them, is there a way to find a scheduling of $n$
tasks so that total time to completion is minimized. The work of~\cite{mnich2015scheduling}
shows that there is a FPT-algorithm to solve this problem
when the problem is parameterized by $p_{max}$, the task with longest
execution time.

Due to the similarities between these two problems, we can show that makespan
scheduling can trivially be reduced to DAG makespan scheduling, meaning
that DAG makespan is more difficult to solve. To show this,
we create two terminal vertices $src$ and $dst$ with 0 execution time. For each
task, we create a vertex $u$ with weight equal to its execution time. For each
vertex $u$, we add an edge between $(src,u)$ and one edge between $(u,dst)$.
When this is done, we have a DAG.
This implies that if we have a solver to solve any makespan minimized
scheduling on DAG, we could also apply it to the makespan scheduling problem
without dependencies.

\section{Special Cases (and potential solutions?)}
There are two special cases of this problem that could be solved trivially:
Case 1: $k = 1$ or $k >= |treewidth|$ \\
Case 2: All tasks are dependent on each other (form a line from $src$ to $dst$)
or there are no dependencies at all (makespan scheduling).

\bibliographystyle{unsrt}
\bibliography{reference}
\end{document}
