\documentclass{article}
\usepackage{algorithm, algpseudocode}
\usepackage{amsmath, amssymb, amsthm}
\usepackage{color}
\usepackage{enumerate}
\usepackage{geometry}
\usepackage{graphicx}
\usepackage{hyperref}
\usepackage{fancyhdr,lastpage}
\usepackage{url}
\usepackage{tabularx}
\pagestyle{fancy}

% Create theorem environments as needed
\newtheorem{theorem}{Theorem}
\newtheorem{definition}{Definition}

\title{Team 2 Opporotunity Identification Project}
\author{Lingnan Gao, Andrew Wright}
\date{October 2017}


\begin{document}
\maketitle
\begin{abstract}
	content...
\end{abstract}
\newcommand{\defproblem}[4]{%
  \hfill\\\smallskip\noindent%
  \begin{tabularx}{\textwidth}{|l X|}%
    \hline%
    \multicolumn{2}{|l|}{{#1}}\\%
    \textbf{Input:}&#2\\%
%    \textbf{Parameter:}&#3\\%
    \textbf{Question:}&#4\smallskip\\\hline%
  \end{tabularx}%
  \smallskip%
}%
\section{Introduction}
Scheduling on Directed Acyclic Graph (DAG) is a crucial problem whose 
applications can be found in many ways, such task scheduling in parallel 
systems and workflow distributions. Typically, an optimal scheduling 
would on a DAG would lead to a minimized completion time,
reduced costs. etc. Due to the significance of this problem, it has been 
extensively studies. Challenges arise when it comes to find an optimal solution 
to this problem, as this problem is NP-hard, finding an optimal solution may 
often be computationally prohibitive to find. 

Although many works have been carried out to find an solution to this, most 
of those work rely on heuristic methods (e.g. HEFT \cite{topcuoglu2002performance}) or 
meta-heuristics (e.g. generic algorithms) to solve this problem. 
While those problems may work well under some conditions, 
it cannot always promise to find an efficient answer.  

One alternative solution is to come up with an Fixed Parameter Tractable 
(FPT)
algorithm to solve this problem. However, to the best of our knowledge,
there is no FPT available 
no FPT-hardness (?) result is given for this problem.

\section{Open Problem Description}
The open problem we identified is makespan scheduling on a DAG.
Given a weighted Directed Acyclic Graph $G=(V,E)$, the vertex
$v \in V$ denotes a task to be executed, while the weight on $v$ 
stands for the execution time. A directed edge on $G$ reflects
dependencies among tasks, an edge from vertex $u$ to $v$
means task $v$ cannot be started unless $u$ has been processed. 
Our objective is, given M identical machines, we want find out the execution
orders for those tasks in order minimize the time to completion.

\defproblem{Makespan scheduling on Directed Acyclic Graph}%
{A directed graph $G = \{V, E\}$ with a function $W_v: V \to \mathbb{R}$,
 where $W_v$ is vertex weights, and a set of identical machines,
$M$}%
{}%
{Is there a Fixed Parameter Tractable algorithm parameterized by tree width
which  minimizes the makespan while respecting
dependencies?}%

A candidate to be used as a parameter is M, the total number of machines
we have. Another parameter we would consider would be treewidth.

\section{Related Questions Works}
To the best our knowledge, the FPT hardness result and solutions to the 
makespan scheduling for DAG remains still open. 
However, we observer that this problems is closely related to makespan scheduling.

For a makespan scheduling problem is that for $n$ weighted tasks with no
dependencies among them, is there any ways to find a scheduling of $n$
tasks so that total time to completion is minimized. The work of~\cite{mnich2015scheduling}
shows that there is a FPT-algorithm to solve this problem within $O(?)$ time,
when this problem is parameterized by the $p_{max}$, the task with longest
execution time.

Due to the similarities between these two problems, we can show that make
span scheduling can trivially be reduced to DAG makespan scheduling, meaning
that DAG makespan is more difficult to solve:
We create two terminal vertices $src$ and $dst$ with 0 execution time. For each
task, we create a vertice $u$ with weight equal to its execution time. For each
vertex $u$, we add an edge between $(src,u)$ and one edge between $(u,dst)$. 
When this step is done, we have a DAG. 
This implies that if we have a solver to solve any makespan
scheduling on DAG, we could makespan scheduling problem. 

\section{Special Cases (and potential solutions?)}
There are two special cases of this problem that could be solved trivially:
Case 1: $k = 1$ or $k >= |treewidth|$ \\
Case 2: All tasks are dependent on each other (form a line from $src$ to $dst$) 
or there is no dependcies at all (makespan scheduling).

\bibliographystyle{unsrt}
\bibliography{reference}
\end{document}
