% Template file for an a0 landscape poster.
% Written by Graeme, 2001-03 based on Norman's original microlensing
% poster.
%
% See discussion and documentation at
% <http://www.astro.gla.ac.uk/users/norman/docs/posters/>
%
% $Id: poster-template-landscape.tex,v 1.2 2002/12/03 11:25:46 norman Exp $


% Default mode is landscape, which is what we want, however dvips and
% a0poster do not quite do the right thing, so we end up with text in
% landscape style (wide and short) down a portrait page (narrow and
% long). Printing this onto the a0 printer chops the right hand edge.
% However, 'psnup' can save the day, reorienting the text so that the
% poster prints lengthways down an a0 portrait bounding box.
%
% 'psnup -w85cm -h119cm -f poster_from_dvips.ps poster_in_landscape.ps'

\documentclass[a0]{a0poster}
% You might find the 'draft' option to a0 poster useful if you have
% lots of graphics, because they can take some time to process and
% display. (\documentclass[a0,draft]{a0poster})

\pagestyle{empty}
\setcounter{secnumdepth}{0}
\setlength\paperwidth{24in}
\setlength\paperheight{36in}
\setlength\textwidth{\paperwidth}
\setlength\textheight{\paperheight}

% The textpos package is necessary to position textblocks at arbitary
% places on the page.
\usepackage[overlay,absolute]{textpos}
\usepackage{enumitem}
\setlist[itemize]{noitemsep,nolistsep}
% Graphics to include graphics. Times is nice on posters, but you
% might want to switch it off and go for CMR fonts.
\usepackage{graphicx}
\usepackage{amsmath}
\usepackage{color}
\usepackage{wallpaper}
\usepackage{tikz}
\usepackage{url}
\usepackage{varwidth}
\usepackage{multicol}
\usepackage{hyperref}
\usetikzlibrary{arrows,automata,positioning,shapes,fit}


% These colours are tried and tested for titles and headers. Don't
% over use color!

\definecolor{DarkBlue}{rgb}{0.1,0.1,0.5}
\definecolor{Red}{rgb}{0.9,0.0,0.1}
\definecolor{Brown}{rgb}{0.4,0.2,0.01}
\definecolor{LightGray}{rgb}{0.9,0.9,0.9}
\definecolor{Sage}{rgb}{0.45,0.9,0.675}
\definecolor{Yellow}{rgb}{1,.95,.5}
\definecolor{Gold}{rgb}{1,.84,0}
\definecolor{mygold}{RGB}{255,204,0}
\definecolor{chocolate}{RGB}{97,51,24}
\definecolor{mygreen}{RGB}{173,214,50}

\definecolor{WolfpackRed}{RGB}{204,0,0}

\definecolor{background}{RGB}{255,255,255}
\definecolor{white}{RGB}{255,255,255}
\pagecolor{background}

% generate a gradient image using http://www.generateit.net/gradient/
%   used 1440x1920 for gradient res from #FFFFFF to #D6E8FF
%   our old color was #B0D5FF
%\TileWallPaper{40in}{30in}{gradient.png}

\color{black}


\definecolor{theorem1}{rgb}{0.154,0.242,0.883}
\definecolor{corollary1}{rgb}{0.084,0.554,0.199}
\definecolor{theorem2}{rgb}{0.478,0.141,0.448}
\definecolor{theorem3}{rgb}{0.189,0.861,0.372}
\definecolor{theorem4}{rgb}{0.925,0.123,0.138}

\newcommand{\smallspace}{\vspace{2mm}}

% see documentation for a0poster class for the size options here
\let\Textsize\normalsize

\def\Head#1{\noindent\hbox to \hsize{\hfil{\Huge\color{chocolate} #1}}\bigskip}
\def\LHead#1{\colorbox{WolfpackRed}{\noindent{\textbf{\Large\color{white} #1}}\bigskip}}
\def\LHeadtwo#1{\noindent{\LARGE\color{white}
#1}\bigskip}
\def\LHeadthree#1{\noindent{\Huge\color{chocolate}
#1}\bigskip}
\def\Subhead#1{\noindent{\LARGE\color{black} #1}\bigskip}
\def\Title#1{\noindent{\textbf{\VeryHuge\color{DarkBlue} #1}}}
\def\Subtitle#1{\noindent{\textbf{\Huge\color{DarkBlue} #1}}}
\def\RedFont#1{\noindent{\textbf{\normalsize\color{WolfpackRed} #1}}}
\def\LHeadfour#1{\colorbox{theorem1}{\noindent{\textbf{\large\color{white} #1}}\bigskip}}
\def\LHeadfive#1{\colorbox{corollary1}{\noindent{\textbf{\large\color{white} #1}}\bigskip}}
\def\LHeadsix#1{\colorbox{theorem2}{\noindent{\textbf{\large\color{white} #1}}\bigskip}}
\def\LHeadseven#1{\colorbox{theorem3}{\noindent{\textbf{\large\color{white} #1}}\bigskip}}
\def\LHeadeight#1{\colorbox{theorem4}{\noindent{\textbf{\large\color{white} #1}}\bigskip}}


% Set up the grid
%
% Note that [40mm,40mm] is the margin round the edge of the page --
% it is _not_ the grid size. That is always defined as
% PAGE_WIDTH/HGRID and PAGE_HEIGHT/VGRID. In this case we use
% 23 x 12. This gives us three columns of width 7 boxes, with a gap of
% width 1 in between them. 12 vertical boxes is a good number to work
% with.
%
% Note however that texblocks can be positioned fractionally as well,
% so really any convenient grid size can be used.
%

\usepackage{colortbl}
\usepackage{multirow,bigdelim}
\usepackage{arydshln}

\definecolor{LightCyan}{rgb}{0.88,1,1}




\begin{document}


\TPGrid[0.5in,0.5in]{23}{35}

\parindent=0pt
\parskip=0.5\baselineskip

% for background gradient


% Understanding textblocks is the key to being able to do a poster in
% LaTeX. In
%
%    \begin{textblock}{wid}(x,y)
%    ...
%    \end{textblock}
%
% the first argument gives the block width in units of the grid
% cells specified above in \TPGrid; the second gives the (x,y)
% position on the grid, with the y axis pointing down.

% You will have to do a lot of previewing to get everything in the
% right place.

% This gives good title positioning for a portrait poster.
% Watch out for hyphenation in titles - LaTeX will do it
% but it looks awful.

\begin{textblock}{24}(0,0)
\begin{center}
\vspace{-0.25in}
\huge{\textbf{FPT Approach to Minimized Makespan Scheduling with Dependencies}}\\
\vspace{0.35in}
\large{\textbf{Lingnan Gao and Andrew Wright}}
 \end{center}
\end{textblock}

% NC State logo brick to left of title
% \begin{textblock}{6}(0,0)
% \begin{center}
% \includegraphics[width=4in]{images/ncsu.png}
% \end{center}
% \end{textblock}



%***************************************************************
%                    Problem
%***************************************************************

%This is a full-width textblock
\begin{textblock}{22.5}(0,2.75)
% This is a header
% We defined it above to use the Wolfpack Red color
% To make sure all headers are the same height, we hid two "Q"s in the header
% We hid them by coloring them with Wolfpack Red
\LHead{\parbox{22.43in}{\begin{center}\textcolor{WolfpackRed}{Q}Problem\textcolor{WolfpackRed}{Q}\end{center}}}\\ %*2*

\vspace{0.35in}
% We're inputting content about graphs in general
\begin{multicols}{3}
\Large {\bf Scheduling for Tasks with Dependencies}

\large Scheduling on Directed Acyclic Graphs (DAG) is a crucial problem with
applications in areas such as task scheduling, parallel processing, and
workflow distributions. We propose the search for a Fixed Parameter Tractable
(FPT) algorithm for scheduling with dependencies. The problem as stated is:

\defproblem{Makespan scheduling on Directed Acyclic Graph}%
{A directed graph $G = \{V, E\}$ with a function $W_v: V \to \mathbb{R}$,
 where $W_v$ is vertex weights, and a set of identical machines,
$M$}%
{Tree width of $G = k$}%
{Is there a Fixed Parameter Tractable algorithm parameterized by tree width
which  minimizes the makespan while respecting
dependencies?}%

\large Given a s set of machines and dependent task

\begin{figure}[h]
	\includegraphics*[width=\linewidth]{images/Fig1.png}
\end{figure}

What is an optimal assignment of tasks to minimize the time to execute all of
the tasks?

An optimal assignment to minimize makespan:

\begin{figure}[h]
	\includegraphics*[width=\linewidth]{images/Fig2.png}
\end{figure}

\end{multicols}

\end{textblock}

\begin{textblock}{22.5}(0,17)


\LHead{\parbox{22.43in}{\begin{center}\textcolor{WolfpackRed}{Q}Definitions\textcolor{WolfpackRed}{Q}\end{center}}}\\ %*2*
\vspace{0.25in}
\Large
\textbf{Directed Acyclic Graph (DAG)}

Graph consisting of a set of directed edges with
no cycles

\textbf{Fixed Parameter Tractable (FPT)}

A problem is FPT if when parameterized by
parameter $k$, there is an algorithm with runtime $f(k) n^{O(1)}$.

\textbf{Makespan}: The total amount of time required for a set of tasks to complete,
starting at the beginning of the first task, ending with the completion of the
last.



\end{textblock}

%***************************************************************
%                   Related Work
%***************************************************************
%This is a block on the left
\begin{textblock}{11.25}(0,20)
\LHead{\parbox{11.43in}{\begin{center}\textcolor{WolfpackRed}{Q}Related Work\textcolor{WolfpackRed}{Q}\end{center}}}\\ %*2*

\vspace{0.35in}

% Again we're inputting from a section file
% Remember to make the text large again if you changed it
% (Above we changed the text to Large)
To the best of our knowledge, FPT hardness results and solutions to
minimized makespan scheduling still remain open problems, however, we observe
that this problem is closely related to the problem of makespan scheduling
without dependencies, which has been studied in the work of


\end{textblock}

%***************************************************************
%                   Research Ideas
%***************************************************************

%This places a block on the right hand side
\begin{textblock}{11.25}(11.75,24)
\LHead{\parbox{11.23in}{\begin{center}\textcolor{WolfpackRed}{Q}Research Ideas\textcolor{WolfpackRed}{Q}\end{center}}}\\ %*2*

\Large

\textbf{Step 1:}  Find suitable parameters. \\
Treewidth; Number of Machines; \\
Execution Time for Single Task
 
\textbf{Step 2:} Consider Special Cases:\\
1) one machine \\
2) $|M|\geq$ maximum cut \\
3) No dependencies

\textbf{Step 3:} Consider Special DAG Tologies.\\
1) one ''line" 2) Tree 3) Grids

\textbf{Step 4:} Find a generalized FTP algorithms.

\bibliographystyle{unsrt}
\bibliography{reference}
\end{textblock}





\end{document}
